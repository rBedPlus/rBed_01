Summary

Several important findings observed through these three experiments. Firstly, results from the PackFlix and Maximum Matching experiments depict how important of choosing efficient data structure smartly affects the asymptotic experiments, especially for large dataset. It is slow using classical Java data structures, like Map ADTs, to pair each key and value. Even worse, reading input files and converting to each object are extremely time-consuming in Java. However, data.table package in R can efficiently solve this problem with only few lines of code. In Figure 2, comparing the total runtime between Java_chainHash and R_dataTable, we can observe that using data table can produce around five times faster result than using Java Map ADT. And in Figure 5, the matching problem can be solved faster in R than Java. All these results show the high efficiency of data.table in R. Then, there is a following question for Java developers: Is there a way to improve the efficiency in these asymptotic experiments with P and NP problems? The second observation in this paper is how stochastic heuristic improves the greedy algorithm. In our last experiment, we used Chvatal algorithm to solve Set Covering Problem. However, we found the limitation of Chvatal algorithm, in which the performance depends on the order of the columns selected. Then, we introduced the concepts of Isomorphs and and isomorphic heuristic. With our efforts and improvements, our isomorph version Chvatal algorithm performs better than the basic version of Chvatal algorithm. Our last observation corresponds to the theorem from Chvatal. In paper OPUS-setc-1979-OR-Chvatal-greedy, Chvatal stated that the cost of the cover returned by his greedy heuristic is at most H(d) times the cost of an optimal cover, where H(d) is the harmonic number of the largest column degree. However, by going through the UB column in Table 2, his theorem does not give a precise upper bound, instead, his theorem only performs good in chvatal_6_5.cnfW, which the instance was generated by Chvatal on purpose.

Conclusion

Several conclusions can be made by these three experiments. Firstly, when analyzing large dataset, R is a great tool with efficient packages such as data.table. Standard Java libraries may not be suitable for data analyzing or machine learning tasks. Secondly, in Chvatal’s algorithm, and many other greedy algorithms, the stochastic implementation can usually improve the performance. However, to fully understand how it affects the results, we need to use the concept of isomorph to determine what really affects the performance for the problem. Lastly, based on the theorem of Chvatal’s algorithm, we conclude that a good heuristic relies on BOTH column degrees AND weights, while his bound relies on harmonic sum only and max(colDegrees), but NOT on weights. 